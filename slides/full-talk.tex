\documentclass{beamer}

\usepackage{beamerthemesplit}
\usepackage{graphicx}
\usepackage{color, natbib, hyperref}
\usepackage{bibentry}
\usepackage[export]{adjustbox}% http://ctan.org/pkg/adjustbox
\nobibliography*

% define colors
\definecolor{jblue}  {RGB}{20,50,100}
\definecolor{ngreen} {RGB}{98,158,31}

%theme

\usetheme{boxes} 
%\usecolortheme{seahorse} 
\setbeamertemplate{items}[default] 
%\setbeamercovered{transparent}
\setbeamertemplate{blocks}[rounded]
\setbeamertemplate{navigation symbols}{} 
% set the basic colors
\setbeamercolor{palette primary}   {fg=black,bg=white}
\setbeamercolor{palette secondary} {fg=black,bg=white}
\setbeamercolor{palette tertiary}  {bg=jblue,fg=white}
\setbeamercolor{palette quaternary}{fg=black,bg=white}
\setbeamercolor{structure}{fg=jblue}
\setbeamercolor{titlelike}{bg=jblue,fg=white}
\setbeamercolor{frametitle}{bg=jblue!10,fg=jblue}
\setbeamercolor{cboxb}{fg=black,bg=jblue}
\setbeamercolor{cboxr}{fg=black,bg=red}

% reduce space before/after equations
\expandafter\def\expandafter\normalsize\expandafter{%
    \normalsize
    \setlength\abovedisplayskip{1pt}
    \setlength\belowdisplayskip{1pt}
    \setlength\abovedisplayshortskip{1pt}
    \setlength\belowdisplayshortskip{1pt}
}

% set colors for itemize/enumerate
\setbeamercolor{item}{fg=ngreen}
\setbeamercolor{item projected}{fg=white,bg=ngreen}

% set colors for blocks
\setbeamercolor{block title}{fg=ngreen,bg=white}
\setbeamercolor{block body}{fg=black,bg=jblue!10}

% set colors for alerted blocks (blocks with frame)
\setbeamercolor{block alerted title}{fg=white,bg=jblue}
\setbeamercolor{block alerted body}{fg=black,bg=jblue!10}
\setbeamercolor{block alerted title}{fg=white,bg=dblue!70} % Colors of the highlighted block titles
\setbeamercolor{block alerted body}{fg=black,bg=dblue!10} % Colors of the body of highlighted blocks

% set the fonts
\usefonttheme{professionalfonts}

\setbeamerfont{section in head/foot}{series=\bfseries}
\setbeamerfont{block title}{series=\bfseries}
\setbeamerfont{block alerted title}{series=\bfseries}
\setbeamerfont{frametitle}{series=\bfseries}
\setbeamerfont{frametitle}{size=\Large}
\setbeamerfont{block body}{series=\mdseries}
\setbeamerfont{caption}{series=\mdseries}
\setbeamerfont{headline}{series=\mdseries}


% set some beamer theme options
\setbeamertemplate{title page}[default][colsep=-4bp,rounded=true]
\setbeamertemplate{sections/subsections in toc}[square]
\setbeamertemplate{items}[circle]
\setbeamertemplate{blocks}[width=0.0]
\beamertemplatenavigationsymbolsempty

% Making a DAG
\usepackage{tkz-graph}  
\usetikzlibrary{shapes.geometric}
\tikzstyle{VertexStyle} = [shape            = ellipse,
                               minimum width    = 6ex,%
                               draw]
 \tikzstyle{EdgeStyle}   = [->,>=stealth']      


% Math macros
\newcommand{\cD}{{\mathcal D}}
\newcommand{\cF}{{\mathcal F}}
\newcommand{\todo}[1]{{\color{red}{TO DO: \sc #1}}}

\newcommand{\reals}{\mathbb{R}}
\newcommand{\integers}{\mathbb{Z}}
\newcommand{\naturals}{\mathbb{N}}
\newcommand{\rationals}{\mathbb{Q}}

\newcommand{\ind}[1]{1_{#1}} % Indicator function
\newcommand{\pr}{\mathbb{P}} % Generic probability
\newcommand{\ex}{\mathbb{E}} % Generic expectation
\newcommand{\var}{\textrm{Var}}
\newcommand{\cov}{\textrm{Cov}}

\newcommand{\normal}{N} % for normal distribution (can probably skip this)
\newcommand{\eps}{\varepsilon}
\newcommand\independent{\protect\mathpalette{\protect\independenT}{\perp}}
\def\independenT#1#2{\mathrel{\rlap{$#1#2$}\mkern2mu{#1#2}}}

\newcommand{\convd}{\stackrel{d}{\longrightarrow}} % convergence in distribution/law/measure
\newcommand{\convp}{\stackrel{P}{\longrightarrow}} % convergence in probability
\newcommand{\convas}{\stackrel{\textrm{a.s.}}{\longrightarrow}} % convergence almost surely

\newcommand{\eqd}{\stackrel{d}{=}} % equal in distribution/law/measure
\newcommand{\argmax}{\arg\!\max}
\newcommand{\argmin}{\arg\!\min}


\mode<presentation>

\title[permute]{permute: A Python Package for Randomization Inference}
\author{Kellie Ottoboni}
\institute[]{Department of Statistics, UC Berkeley\\Berkeley Institute for Data Science}
\date{June 14, 2016}

\begin{document}

\frame{
\titlepage
\vfill
\begin{columns}[T]
\begin{column}{.5\textwidth}
\begin{center}
\vspace{25pt}
\includegraphics[width=\textwidth]{logo/dept1.pdf}
\end{center}
\end{column}
\begin{column}{.3\textwidth}
\end{column}
\begin{column}{.3\textwidth}
\begin{center}
\includegraphics[width=0.9\textwidth]{logo/BIDS.png}
\end{center}
\end{column}
\end{columns}
}

\AtBeginSection[]
{
   \begin{frame}
       \frametitle{Outline}
       \tableofcontents[currentsection]
   \end{frame}
}



\section[Introduction]{Introduction}


\frame{
History of randomization inference
- fisher
- neyman model
}


\frame{
In R:
\begin{itemize}
\item \texttt{ri} by Peter Aronow and Cyrus Samii

{\tiny\textit{``This package provides a set of tools for conducting exact or approximate inference for randomized experiments of arbitrary design. The primary functionality of the package is in the generation, manipulation and use of permutation matrices implied by given experimental designs...''}\par}
\item \texttt{RItools} by Mark Fredrickson

{\tiny\textit{``The RItools package implements useful functions for implementing randomization inference based statistical tests. The package provides tools for testing balance of observed covariates in observational studies using the methodology of:...The package also provides outcome analysis of simple or block randomized trials (or matched observational studies) based on user defined models and test statistics.''}\par}
\item \texttt{coin} by Torsten Hothorn, Kurt Hornik, Mark A. van de Wiel, and Achim Zeileis

{\tiny\textit{The R package coin implements a unified approach to permutation tests providing a huge class of independence tests for nominal, ordered, numeric, and censored data as well as multivariate data at mixed scales. Based on a rich and flexible conceptual framework that embeds different permutation test procedures into a common theory, a computational framework is established in coin that likewise embeds the corresponding R functionality in a common S4 class structure with associated generic functions.}\par}
\item \texttt{perm} by Michael Fay

{\tiny\textit{The package has three main functions, to perform linear permutation tests. These tests are tests where the test statistic is the sum of the product of a covariate (usually group indicator) and the scores.}\par}\end{itemize}
}



\frame{
In Python, statistics packages are limited.

\begin{itemize}
\item \texttt{numpy.random} generates random variables from common distributions
\item \texttt{scipy.stats} computes moments, evaluates distribution functions, and generates random variables from common distributions and does common tests
\item \texttt{StatsModels} is a Python module that provides classes and functions for the estimation of many different statistical models, as well as for conducting statistical tests, and statistical data exploration. 
\item \texttt{scikit-learn} is a module for machine learning 
\end{itemize}
}


\frame{
\frametitle{Python is gaining popularity for doing data analysis}
PYPL Popularity of Programming Language Index, Worldwide

\begin{center}
\begin{figure}[htbp]
\includegraphics[width=\textwidth]{google_trends/pypl.png}
\end{figure}
{Keyword: how to}
\end{center}

}

\frame{
\frametitle{Python is gaining popularity for doing data analysis}
Google trends on May 22, 2016

\begin{center}
\begin{figure}[htbp]
\includegraphics[width=\textwidth]{google_trends/dataanalysis}
\end{figure}
{Keyword: data analysis}
\end{center}

}

\frame{
\frametitle{Python is gaining popularity for doing data analysis}

Google trends on May 22, 2016
\begin{center}
\begin{figure}[htbp]
\includegraphics[width=\textwidth]{google_trends/datascience}
\end{figure}
{Keyword: data science}
\end{center}

}



\section[Examples]{Examples}
\subsection[Teaching evaluations]{Gender bias in teaching evaluations}
\frame{
Data comes from \citet{macnell2014whats}
}


\subsection[Salt and mortality]{Salt and mortality at the level of nations}

\frame{
salt and mortality
}


\subsection[Inter-rater reliability]{Inter-rater reliability}
\frame{
NSGK IRR stuff
\citet{millman2016case}
}

\section[Software and Statistics]{The role of software development in Statistics}

\frame{
Reproducibility crisis:
\begin{itemize}
\item Why Most Published Research Findings Are False (Ioannidis, 2005)
\item 30--50\%\todo{} of studies fail to replicate (\todo{cite})
\end{itemize}
}

\frame{
\textbf{Why?} 

\begin{itemize}
\item File drawer problem
\item Publication bias: positive findings are more likely to get published
\item P-hacking and trying many models before reporting one
\item Inappropriate statistical tests
\end{itemize}

Randomization inference may ameliorate the last problem
}

\frame{
\frametitle{Download \texttt{permute}!}
\begin{figure}[htbp]
\begin{center}
\includegraphics[width=\textwidth]{github/permute}
\end{center}
\end{figure}


\url{https://github.com/statlab/permute}
}


\frame{
\frametitle{Collaborators}
\begin{figure}[htbp]
\begin{center}
\includegraphics[width = 0.3\textwidth, valign=t]{github/jarrodmillman} 
\includegraphics[width = 0.3\textwidth, valign=t]{github/pbstark} 
\includegraphics[width = 0.3\textwidth, valign=t]{github/stefanv} 
\end{center}
\end{figure}

}

\begin{frame}
\frametitle{References}
\tiny
\bibliographystyle{plainnat}
\bibliography{refs}
\itemize
\end{frame}


\end{document}
